\documentclass[animated,a4paper,slidestop,xcolor=pst,blue]{beamer}

\input{slidesHeader.tex}

\title[Métodos de Desarrollo]{Métodos de Desarrollo}

\author[Pablo Sánchez]{\alert{Pablo Sánchez}}

\institute[I2E]{
		   Dpto. Ingeniería Informática y Electrónica \\
		   Universidad de Cantabria \\
		   Santander (Cantabria, España) \\
		   p.sanchez@unican.es
}

\date{}

\begin{document}

\begin{frame}[c]
	\titlepage
	\begin{columns}
		\column{0.50\linewidth}
			\centering
    		\includegraphics[width=.28\textwidth,keepaspectratio=true]{images/istr.eps}
		\column{0.50\linewidth}
			\centering
			\includegraphics[width=.25\textwidth,keepaspectratio=true]{images/uc.eps}
	\end{columns}
\end{frame}

\section{Profesorado}

\begin{frame}[c]
	\frametitle{Profesorado}
	\begin{center}
		\alert{Pablo S\'{a}nchez Barreiro}  \\
		\begin{small}
		Despacho 1069 \\
		Departamento de Ingeniería Informática y Electrónica \\
		p.sanchez@unican.es \\
		\end{small}
        \ \\
		\alert{Julio L. Medina Pasaje}  \\
		\begin{small}
		Despacho 1080 \\
		Departamento de Ingeniería Informática y Electrónica \\
		julio.medina@unican.es \\
		\end{small}
	\end{center}
\end{frame}

\section{Objetivos y Temario}

\subsection{Objetivos}

\begin{frame}[c]
	\frametitle{Objetivos de la Asignatura}
    \begin{enumerate}[<+->]
        \item Conocer y comprender los fundamentos de la gestión de la configuración.
        \item Ser capaz de gestionar configuraciones de productos software.
        \item Conocer el concepto de integración continua.
        \item Conocer la diferencia entre metodologías planificadas y ágiles.
        \item Conocer los fundamentos de las metodologías ágiles.
        \item Saber aplicar una metodología desarrollo ágil.
        \item Saber modelar, analizar y evaluar procesos de desarrollo software.
        \item Conocer una metodología de desarrollo software estandarizada.
        \item Conocer cómo se llevan a cabo los procesos de mantenimiento sw.
        \item Conocer metodologías de desarrollo en la frontera de la investigación.
	 \end{enumerate}
\end{frame}

\subsection{Objetivos de Aprendizaje}

\begin{frame}[c]
	\frametitle{Objetivos de Aprendizaje}
    	\begin{enumerate}[<+->]
            \item El alumno sabrá gestionar versiones de productos software utilizando \alert{Git}.
            \item El alumno sabrá desarrollar aplicaciones utilizando \alert{SCRUM}.
            \item El alumno sabrá modelar procesos de desarrollo sw en \alert{SPEM} utilizando \alert{EPF Composer}.
            \item El alumno conocerá las áreas principales y fundamentos de \alert{Metricav3}.
		\end{enumerate}
\end{frame}

\subsection{Temario}

\begin{frame}[c]
	\frametitle{Temario}
	\begin{enumerate}[<+->]
		\item Gestión de la Configuración.
		\item Metodologías Ágiles.
		\item Modelado de Procesos Sw.
		\item Procesos de Desarrollo Estandarizados.
        \item Metodologías de Desarrollo Innovadoras.
	\end{enumerate}
\end{frame}

\section{Metodología}

\subsection{Plataforma}

\begin{frame}[c]
	\frametitle{Plataforma de Trabajo}
	\begin{itemize}
		\item<1-> La plataforma de trabajo de la asignatura es \emph{moodle}.
		\item<2-> Todas las notificaciones y publicaciones se harán a través de \emph{moodle}.
		\item<3-> \alert{Es obligación del alumno estar atento a las posibles notificaciones y avisos que se realicen a través de \emph{moodle}}.
	\end{itemize}
\end{frame}

\subsection{Actividades}

\begin{frame}
	\frametitle{Clases en Aula}
	\begin{block}{Objetivo}
        Entender los conocimientos teóricos que constituyen la base de las habilidades y destrezas a adquirir al final de la asignatura.
	\end{block}
	\begin{itemize}
		\item<2-> \alert{Sin conocimiento teórico es imposible alcanzar las habilidades prácticas}.
        \item<3-> La asistencia a clases no es obligatoria, \alert{pero si altamente recomendable e incluso necesaria}.
        \item<4-> La asignatura no está diseñada para ser seguida a distancia.		
        \item<5-> Clases magistrales utilizando pizarra y/o transparencias.
		\item<6-> \alert{Las transparencias} son material de apoyo a la docencia, \alert{no son apuntes}.
        \item<7-> Visualización de vídeotutoriales para la utilización de herramientas.
        \item<8-> Por cada tema existirá un itinerario que indicará cómo preparar el tema de forma autónoma.
	\end{itemize}
\end{frame}

\begin{frame}[c]
	\frametitle{Clases en Laboratorio}
	\begin{block}{Objetivo}
        Aplicar los conceptos teóricos aprendidos en las clases de aula a casos de estudio reales, de mediana escala, con el objetivo de desarrollar las competencias deseadas.
	\end{block}
	\begin{itemize}[<+->]
        \item Realización de pequeñas prácticas guiadas y no calificables.
        \item Desarrollo \alert{en grupo} de un proyecto sw utilizando \emph{SCRUM}.
        \item Dicho proyecto se integrará con las asignaturas de \emph{Calidad y Auditoría} y \emph{Procesos de Ingeniería del Software}.
        \item Modelado de una metodología sw en \emph{SPEM}.
	\end{itemize}
\end{frame}

\subsection{Proyecto Integrado}

\begin{frame}[c]
	\frametitle{Proyecto Sw Integrado}
	\begin{itemize}[<+->]
        \item Durante las semanas 5 a 10 se trabajará exclusivamente en el desarrollo de un proyecto sw utilizando \emph{Scrum} como metodología.
        \item No hay clases teóricas, sólo clases de laboratorio.
        \item Los alumnos deberán desarrollar en grupos de 4-6 alumnos una aplicación \emph{Android} para la visualización de una fuente externa de datos.
        \item Además, la asignatura de \emph{Calidad y Auditoría} integra sus cuatro horas con la asignatura de \emph{Métodos de Desarrollo} y \emph{Procesos de Ingeniería del Sw} integra dos horas.
	\end{itemize}
\end{frame}

\begin{frame}[c]
	\frametitle{Proyecto Sw Integrado}
	\begin{itemize}[<+->]
        \item Se utilizarán durante el desarrollo del proyecto conceptos y técnicas de las tres asignaturas.
        \item Se utilizarán diferentes partes del proyecto para evaluar cada asignatura por separado.
        \item La integración permite desarrollar proyectos de mayor envergadura y trabajar como si se hiciese en una empresa real.
        \item La no asistencia a clases durante el proyecto integrado deberá estar aceptada por el resto del grupo de prácticas.
	\end{itemize}
\end{frame}

\subsection{Horarios}

\begin{frame}[c]
	\frametitle{Horario Clases Regular}
	\begin{small}
	\begin{center}
	\begin{tabular}{||l|c|c|c|c|c||}
	\hline \hline
				   & Lunes   & Martes  & Miércoles & Jueves  & Viernes \\ \hline \hline
    08:30 - 09:30  &         &         &  Métodos  & Métodos &         \\ \hline
    09:30 - 10:30  &         &         &           &         &         \\ \hline
    10:45 - 11:45  & Métodos &         &           &         &         \\ \hline
	11:45 - 12:45  & Métodos &         &           &         &         \\ \hline
	12:45 - 13:45  &         &         &           &         &         \\ \hline \hline
	\end{tabular}
	\end{center}
	\end{small}
\end{frame}

\begin{frame}[c]
	\frametitle{Horario Clases Proyecto Integrado}
	\begin{small}
	\begin{center}
	\begin{tabular}{||l|c|c|c|c|c||}
	\hline \hline
				   & Lunes     & Martes   & Miércoles & Jueves   & Viernes \\ \hline \hline
    08:30 - 09:30  & Proyecto  & Proyecto & Proyecto  & Proyecto &         \\ \hline
    09:30 - 10:30  & Proyecto  & Proyecto & Proyecto  & Proyecto &         \\ \hline
    10:45 - 11:45  & Proyecto  &          &           &          &         \\ \hline
	11:45 - 12:45  & Proyecto  &          &           &          &         \\ \hline
	12:45 - 13:45  &           &          &           &          &         \\ \hline \hline
	\end{tabular}
	\end{center}
	\end{small}
\end{frame}

\section{Métodos de Evaluación/Calificación}

\begin{frame}[c]
	\frametitle{Cálculo de la Calificación Final}
	\begin{block}{Fórmula de Cálculo de la Calificación Final}
		\begin{tabular}{lll}
			Calificacion Final = & Scrum & $\times \ 0.70 \ + $ \\
                                 & SPEM  & $\times \ 0.15 \ + $ \\
                                 & Git   & $\times \ 0.15 \   $ \\
		\end{tabular}
	\end{block}
	\begin{itemize}
		\item<2-> Hay que obtener una calificación mínima de 5.00 en la parte de Scrum para poder superar la asignatura.
		\item<3-> Hay que obtener una calificación mínima de 3.50 de media en las otras partes para poder superar la asignatura.
        \item<4-> La parte de Git se evaluará exclusivamente mediante una prueba práctica.
        \item<5-> La parte de SPEM se evaluará mediante una práctica a realizar en casa más una prueba práctica y teórica.
	\end{itemize}
\end{frame}

\section{Bibliografía}

\begin{frame}[c]
    \frametitle{Bibliografía Principal}
    \begin{thebibliography}{1}

\bibitem{Chacon2014}
Scott Chacon and Ben Straub.
\newblock {\em {Pro Git}}.
\newblock Apress, 2 edition, 2014.

\bibitem{Schwaber2011}
K~Schwaber and Jeff Sutherland.
\newblock {The Scrum Guide}.
\newblock {\em Scrum. org, October}, 2(October):17, 2011.

\bibitem{Cohn2004}
Mike Cohn.
\newblock {\em {User Stories Applied: For Agile Software Development}},
\newblock Addison-Wesley Professional, 2004.

\bibitem{}
Object Management~Group (OMG).
\newblock {\em Software \& Systems Process Engineering Meta-Model Specification 2.0}.
\newblock OMG Document number: formal/2008-04-01, April 2008.
\end{thebibliography}
\end{frame}

\begin{frame}[c]
    \frametitle{Bibliografía Secundaria}
    \begin{thebibliography}{1}
\bibitem{sommerville:2010}
Ian Sommerville.
\newblock {\em {Software Engineering}}.
\newblock Addison Wesley, 9 edition, 2010.


\bibitem{Cockburn2006}
Alistair Cockburn.
\newblock {\em {Agile software development: the cooperative game}}
\newblock Addison-Wesley Professional, 2006.

\bibitem{}
Ministerio de Hacienda y Administraciones Públicas.
\newblock Metodología MÉTRICA Versión 3.

\end{thebibliography}
\end{frame}

\end{document}
